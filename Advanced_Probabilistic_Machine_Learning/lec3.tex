%!TEX root = master.tex

\section{Supervised machine learning}
Supervised machine learning involves methods for learning a model for the relationship between the input $x$ and the output $y$ from some observed data set $\mathcal{D} = \{(x_1,y_1), (x_2,y_2), \ldots, (x_N,y_N)\}$. This lecture will cover how we can use what we learned in the course: Statistical machine learning into use with the Bayesian concept. We will do this by introducing: \textbf{Bayesian linear regression}

The model is after training used to \textbf{predict} the output from an unseen input. The question for us is, how to reformulate the supervised machine learning problem into the probabilistic setting?


\subsection{Classic linear regression}
Lets recall the linear regression form the SML course:

\begin{equation}
	y_n = \bm{w}^{T} \bm{x_n} \epsilon_n, \quad \epsilon \sim \mathcal{N}(0,\Sigma^{2}),  
\end{equation}

\begin{itemize}
	\item $y_n$ - observed \textcolor{red}{random} variable
	\item \textbf{w} - unknown \textcolor{blue}{deterministic} variable 
	\item $x_n$ known \textcolor{blue}{deterministic} variable
	\item $\epsilon_n$ - unknown \textcolor{red}{random} variable
	\item $\sigma $ - known \textcolor{blue}{deterministic} variable 
\end{itemize}

\subsection{Linear regression: Maximum likelihood}
There are two equivalent ways of expressing the linear regression model:

\begin{itemize}
	\item $y_n = \bm{w}^{T}x_n + \epsilon_n, \quad \epsilon_n \sim \mathcal{N}(0, \sigma ^{2} )$ 
	\item $p(y_n|\bm{w}) = \mathcal{N}(y_n; \bm{w}^{T}\bm{x}_n, \sigma ^{2}$ 
\end{itemize}

The likelihood $p(y_n|\bm{w})$ is given by:

\begin{equation}
	p(\bm{y}|\bm{w}) = \prod^N_{n=1} p(y_n|w) =  \prod^N_{n=1} \mathcal{N}(y_n; w^{T}x_n, \sigma ^{2}) = \mathcal{N}(\bm{y}; \bm{Xw}, \sigma ^{2}\bm{I}_N)
\end{equation}

The solution is found by maximizing the likelihood: $\hat{w} = \underset{w}{\text{arg max}} p(\bm{y}|w)$ 

\subsection{Bayesian linear regression}
We now introduce a prior over the parameter $w$. Bayesian linear regression model: 

\begin{wbox}{}
\begin{equation}
\begin{aligned}
	y_n = \bm{w}^{T}\bm{x}_n + \epsilon_n, \quad \epsilon \sim \mathcal{N}(0, \sigma ^{2}), \quad n=1,\ldots, N, \\
	\bm{w} \sim p(\bm{w})
\end{aligned}
\end{equation}
\end{wbox}

The present assumptions for the model:
\begin{itemize}
	\item $y_n$ - observed \textcolor{red}{random} variable
	\item \textbf{w} - unknown \textcolor{red}{random} variable 
	\item $x_n$ known \textcolor{blue}{deterministic} variable
	\item $\epsilon_n$ - unknown \textcolor{red}{random} variable
	\item $\sigma $ - known \textcolor{blue}{deterministic} variable 
\end{itemize}

The task is to compute the posterior distribution: $p(w|y)$.

Recall from lecture two, the multivariate Gaussian. For a D-dimensional vector $x$, the multivariate Gaussian distribution can be written on the form:

\begin{equation}
	\mathcal{N}(x; \bm{\mu}, \Sigma) =\underset{Z}{ \frac{1} {(2 \pi) ^{D/2}\sqrt{\text{get}\Sigma}} } \text{exp} \left( - \frac{1} {2} (x- \bm{\mu })^{T}\Sigma ^{-1}(x- \bm{\mu }) \right)  
\end{equation}

\begin{itemize}
	\item $\bm{\mu }$ is the mean vector
	\item $\Sigma$ is the co variance matrix 
	\item Z is the normalization constant 
\end{itemize}

\begin{wbox}{}
The probabilistic model is given by:

\begin{equation}
\begin{aligned}
	pp(y|w) = \mathcal{N}(y;\bm{Xw}, \beta ^{-1}\bm{I}_N), \quad \text{likelihood} \\
	p(w)= \mathcal{N}(w; \bm{m}_0, S_0), \quad \text{prior distribution}
\end{aligned}
\end{equation}
\end{wbox}

The task is to compute the posterior distribution: $p(w|y)$. The solution is to identify:
\begin{equation}
	x_a = w, \quad x_b = y
\end{equation}

We use \text{Corollary 1} to get the posterior distribution:
\begin{equation}
	pp(w|y)= \mathcal{N}(w,\bm{m}_N, \bm{S}_n)
\end{equation}


where:
\begin{equation}
\begin{aligned}
	\bm{m}_N = \bm{S}_N (\bm{S}^{-1}\bm{m}_0 + \beta \bm{X}^{T}\bm{y)} \\
	\bm{S}^{-1} = \bm{S}_0^{-1} + \beta \bm{X}^{T}\bm{X}
\end{aligned}
\end{equation}

\begin{example}{Example: Bayesian linear regression}
Consider the problem of fitting a straight line to noisy measurements. We let the model be : $y_n \in \mathbb{R}, x_n \in \mathbb{R}$ 

\begin{equation}
	y_n = w_0 + w_1 x_n + \epsilon_n), \quad n=1, \ldots, N
\end{equation}
where:
\begin{equation}
\begin{aligned}
	x_n = \begin{bmatrix} 1 \\ x_n \end{bmatrix}, \quad w = \begin{bmatrix} w_0 \\ w_1 \end{bmatrix} \\
	\epsilon _n \sim \mathcal{N}(w | (0,0) ^{T}, \alpha ^{-1} \bm{I}_2), \quad \text{where } \alpha = 2
\end{aligned}
\end{equation}

\begin{itemize}
	\item The prior : $p(w) = \mathcal{N}(w|\begin{pmatrix} 0& 0 \end{pmatrix} ^{T}, \frac{1} {2} \bm{I}_2$ 
	\item The likelihood function: $p(y_2|w) = \mathcal{N}(y_n|w_0+w_1x_2, \beta ^{-1}$
	\item Posterior/prior: $p(w|y_2) = \mathcal{N}(w|\bm{m}_2,$  
 	\item $m_3 = \beta  \bm{S}_3\bm{X}^{T}\bm{y}$
	\item $\bm{S}_3 = (\alpha \bm{I}_2 + \beta \bm{X}^{T}\bm{X}^{-1}$ 

\end{itemize}

\textbf{Predictive distribution}, for a new data point $(y_*, x_*)$  we have:

\begin{equation}
\begin{aligned}
 	p(y_x|w)= \mathcal{N}(y_*; \bm{x}_* ^{T}, \beta ^{-1}), \quad \text{likelihood} \\
	p(w|y) = \mathcal{N}(w;\bm{m}_N, \bm{S}_N), \quad \text{posterior}
\end{aligned}
\end{equation}

This can be applied to multiple types of regression, such as:

\begin{itemize}
	\item Linear: \\
		\begin{equation}
			f(x) = \bm{x}^{T}\bm{w}, \quad \bm{x}= \begin{bmatrix} 1,& x \end{bmatrix}^{T}
		\end{equation}
	\item Quadratic regression: \\
		\begin{equation}
			f(x) = \bm{x}^{T}\bm{w}, \quad \bm{x}= \begin{bmatrix} 1,& x,& x^{2} \end{bmatrix}
		\end{equation}i
	\item Switch regression: \\
		\begin{equation}
			f(x) = \bm{x}^{T}\bm{w}, \quad \bm{x}^{T}= \begin{bmatrix} h(x-S)& h(x-6)& \ldots& h(x+s) \end{bmatrix}, \quad h(x) = 
		\end{equation}
\end{itemize}


\end{example}		

\subsection{Hyperparameters}
We have seen a few examples of how to find the posterior for a parameter $\bm{\theta}$ 

\begin{equation}
	p_\xi ( \theta |\mathcal{D}) = \frac{p_\xi (\mathcal{D}|} {\bm{ \theta } p_\xi (\mathcal{D})} 
\end{equation}

In probabilistic models we usually have \textbf{hyperparameters} $\xi $, for example:
\begin{itemize}
	\item Coin flip example: $\xi = \{a,b\} $, the parameters of the beta prior.
	\item Bayesian linear regression: $\xi = \{ a,b\} $, precision of prior and the likelihood.
\end{itemize}

But how to choose these hyperparameters? We have some alternatives:

\begin{enumerate}
	\item Pick hyperparameters that reflects any prior knowledge of the problem.
	\item The go-to solution for machine learning: $k$-fold cross validation.
	\item The Bayesian alternative: Maximize the marginal likelihood.
	\item And the even more Bayesian alternative: Put priors on the hyperparameters and do inference. 
\end{enumerate}

Since the cross validation alternative was covered in the last course we will only take a look at the two last options.

\subsection{Bayesian approach 1: Marginal likelihood}
The marginal likelihood/evidence $p_\xi (\mathcal{D)}$ says how probable the data $\mathcal{D}$ is in view of the hyperparameters $\xi $. With a similar argument as for the maximum likelihood idea that was presented in the SML course, we can select $\xi $  as :

\begin{equation}
	\hat{\xi } = \text{arg } \underset{\xi }{\text{max}} p_\xi (\mathcal{D})   
\end{equation}

\begin{wbox}{}
Maximizing the likelihood often leads to overfit. \\
Maximizing the \emph{marginal} likelihood rarely leads to overfit, (fewer parameters)
\end{wbox}

To maximize the marginal likelihood is sometimes referred to as the \emph{emperical Bayes}. We have to use numerical optimization, such as BFGS or similar to to optimize the marginal likelihood. The \textbf{idea} is to compute the gradient $\nabla_\xi p(y)$ and numerically estimate the Hessian $\nabla ^{2}_\xi p(y)$. Can be done with newtons methods, one step: $\xi ^{x+1} \leftarrow \xi ^{t}- |\nabla ^{2}_\xi p(y)|^{-1}	|\nabla_\xi p(y)|$. \textbf{Note} that it is important how you initialize the hyperparameter search. 

\subsection{Bayesian approach 2}
The probabilistic model with the unknown $w$ is given by:
\begin{equation}
\begin{aligned}
	p(w) = \mathcal{N(w;\bm{m}_0, \alpha ^{-1}\bm{I}_D}) \\
	p(y|w) = \mathcal{N}(y; \bm{Xw}, \beta ^{-1} \bm{I}_N)
\end{aligned}
\end{equation}

which gives the posterior:

\begin{equation}
p(w|y)= \mathcal{N}(w;\bm{m}_N,S_N)
\end{equation}

Note here that using a Gaussian prior gives a Gaussian posterior. Hence the Gaussian prior is a \textcolor{red}{Conjugate prior} for the Gaussian likelihood with an unknown $w$. A question we might ask here is: \emph{What if also $\beta$ precision is unknown}.

The probabilistic model with unknown $w$ and $ \beta $ is given by:

\begin{equation}
\begin{aligned}
	pp(w| \beta ) = \mathcal{N}(w; \mathbb{m}_0, \beta ^{-1} \alpha ^{-1} \bm{I}_N) \text{Gam} ( \beta; a_0, b_0), \quad \text{prior} \\
	pp(y|w) = \mathcal{N}(y;\bm{Xw}, \beta ^{-1} \bm{I}_N), \quad \text{likelihood}
\end{aligned}
\end{equation}

which will give us the posterior:

\begin{equation}
	p(w, \beta |y)=\mathcal{N}(w;\bm{m}_N, \beta ^{-1}\bm{S}_N) \text{Gam}( \beta , a_N, b_N), \quad \text{posterior}
\end{equation}

Using a Gauss-Gamma prior gives a Gauss-Gamma posterior. Hence the Gauss-Camma prior is a conjugate prior to the Gaussian likelihood with unknown $w$ and unknown precision $ \beta $. For more information: look at exercise 2.11.

\subsection{Summary of lecture 3}
In this lecture we have introduced the Bayesian linear regression and:
\begin{itemize}
	\item How to learn the model
	\item How to use the model for making predictions
	\item How to use features
\end{itemize}

We also presented two Bayesian methods for selecting hyperparameters in a probabilistic model by:
\begin{itemize}
	\item Maximizing marginal likelihood
	\item Using hyperpriors
\end{itemize}









