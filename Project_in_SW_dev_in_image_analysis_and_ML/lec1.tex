%!TEX root = master.tex

\section{Requirement engineering}
Requirements are about \textbf{What} not \textbf{How}. The \textbf{definition}: Requirements are a specification of what should be implemented. They are descriptions of how a system should behave, or of a system property or attribute. They may be a constraint on the development process of the system. 

\subsection{Customers}
\textbf{Stakeholder} is a person, group or organization that: 
\begin{itemize}
	\item is actively  working in a project
	\item is affected by the process or outcome
	\item or can influence the outcome
	\item Stakeholders can be internal or external to the project team and to developing organization.
\end{itemize}

\textbf{A customer} is an individual or organization that derives either direct or indirect benefit from the product that is being developed. 

The \textbf{user} are a subset of the customer who will actually use the product. We can distinguish two types of these:
\begin{itemize}
	\item Direct users that will operate the product hands-on
	\item Indirect users might receive outputs from the system without touching it/come in contact themselves.
\end{itemize}

\subsection{User vs System requirements}
\textbf{User requirements }are goals or tasks that \textcolor{blue}{specific classes of users} \textcolor{red}{must} be able to perform with a system, or a \textcolor{red}{desired} product attribute. 

\textbf{System requirements} are Top-level requirements for a product and can be software or software + hardware.

The user requirements are usually \emph{abstract, in natural language and what the customer wants} while the system requirements are more \emph{concrete and detailed, Natural+formal language, what system provides and is used as a product description }

\begin{example}{A software example}
User requirements:
\begin{itemize}
	\item I need to print a mailing label for a package
	\item As the lead machine operator, I need to calibrate the pump controller first thing every morning
\end{itemize}

System requirements:
\begin{itemize}
	\item If the pressure exceeds 40.0 psi	the high pressure warning light should come on.
	\item The user must be able to sort the project list in forward and reverse alphabetical order. 
\end{itemize}
\end{example}	



\subsection{User requirements}
Written as the use cases/user stories (agile). Informed by 0. Domain requirements: "everyone knows that...". 
\begin{itemize}
	\item \textbf{Use case} 
		\begin{itemize}
			\item Sequence of interactions between a system and external actor to achieve an outcome of value. 
		\end{itemize}
	\item \textbf{User Story} 
		\begin{itemize}
			\item Short, simple description of a feature told from the perspective of who wants it, e.g., user or customer from the system.
			\item As a \textcolor{green}{USER ROLE}, I want \textcolor{blue}{DO SOMETHING} so that \textcolor{red}{REASON}  	
		\end{itemize}
\end{itemize}


\begin{itemize}
	\item \textbf{User classes}
		\begin{itemize}
			\item Subset of product's customer
			\item An individual can belong to several classes
			\item Each user class must have a set of requirements 
			\item Classes need to be human beings
		\end{itemize} 
	\item \textbf{What process to use}
		\begin{itemize}
			\item User personas - A description of a representative of a user class with similar characteristics and needs
			\item User representatives - A suitable representative to provide the voice of the user.
			\item Product Champion - An intermediary gathering requirements form the users for us.
		\end{itemize}
\end{itemize}

In \emph{agile} methods all user classes are represented by the \textcolor{blue}{product owner} 

\subsection{System requirements}
We can divide it into two categories: functional and non-functional:

\textbf{Functional Requirements}
\begin{itemize}
	\item Behavior that a system will exhibit under specific conditions
	\item Describes \textcolor{blue}{what} developers must implement to satisfy under requirements
	\item Written as "shall" statements
\end{itemize}

\begin{example}{Example: Functional system requirements}
	\begin{itemize}
		\item The passenger \textcolor{blue}{shall be able to} print boarding passes for all flight segments for which he has checked in.
		\item If the passenger's profile does not indicate a seating preference, the reservation system \textcolor{blue}{shall} assign a set.
	\end{itemize}
\end{example}	 

\textbf{Non-functional Requirements}
\begin{itemize}
	\item Describe \textcolor{blue}{how well} a system does what it must do
	\item Quality standard of the system
	\item Must be measurable
\end{itemize}

Here are some \textcolor{blue}{metrics} for \textbf{Non-functional System Requirements}

\begin{table}[ht!]
    \centering
    \begin{tabular}{|l|l|}
        \hline
        \textbf{Property} & \textbf{Measure} \\
        \hline
        Speed & Processed transactions/second \\
             & User/event response time \\
             & Screen refresh time \\
        \hline
        Size & Mbytes \\
             & Number of ROM chips \\
        \hline
        Ease of use & Training time \\
                    & Number of help frames \\
        \hline
        Reliability & Mean time to failure \\
                    & Probability of unavailability \\
                    & Rate of failure occurrence \\
                    & Availability \\
        \hline
        Robustness & Time to restart after failure \\
                   & Percentage of events causing failure \\
                   & Probability of data corruption on failure \\
        \hline
        Portability & Percentage of target dependent statements \\
                    & Number of target systems \\
        \hline
    \end{tabular}
    \caption{Example}
    \label{tab:tab1}
\end{table}






